\documentclass[a4paper]{book}
\usepackage[times,inconsolata,hyper]{Rd}
\usepackage{makeidx}
\usepackage[utf8]{inputenc} % @SET ENCODING@
% \usepackage{graphicx} % @USE GRAPHICX@
\makeindex{}
\begin{document}
\chapter*{}
\begin{center}
{\textbf{\huge Package `BackToSchool'}}
\par\bigskip{\large \today}
\end{center}
\begin{description}
\raggedright{}
\inputencoding{utf8}
\item[Title]\AsIs{Implement agent-based model of COVID-19 in elementary schools}
\item[Version]\AsIs{0.0.0.9000}
\item[Imports]\AsIs{tidyverse, igraph}
\item[Description]\AsIs{This package allows users to run an agent-based model of COVID-19 transmission in elementary schools, customized to a particular setting.}
\item[License]\AsIs{`use\_mit\_license()`, `use\_gpl3\_license()` or friends to
pick a license}
\item[Encoding]\AsIs{UTF-8}
\item[LazyData]\AsIs{true}
\item[Roxygen]\AsIs{list(markdown = TRUE)}
\item[RoxygenNote]\AsIs{7.1.0.9000}
\end{description}
\Rdcontents{\R{} topics documented:}
\inputencoding{utf8}
\HeaderA{initialize\_school}{Initialize school}{initialize.Rul.school}
%
\begin{Description}\relax
This function takes in a data frame exported by make\_school().
It adds epidemiological attributes of the full school community.
\end{Description}
%
\begin{Usage}
\begin{verbatim}
initialize_school(
  n_contacts = 10,
  n_contacts_brief = 0,
  rel_trans_HH = 1,
  rel_trans = 1/8,
  rel_trans_brief = 1/50,
  p_asymp_adult = 0.35,
  p_asymp_child = 0.7,
  attack = 0.01,
  child_trans = 1,
  child_susp = 1/3,
  teacher_trans = 1,
  teacher_susp = 1,
  disperse_transmission = T,
  isolate = 1,
  dedens = F,
  run_specials = F,
  start
)
\end{verbatim}
\end{Usage}
%
\begin{Arguments}
\begin{ldescription}
\item[\code{n\_contacts}] Number of sustained contacts outside of the classroom; defaults to 10

\item[\code{n\_contacts\_brief}] Number of brief contacts outside of the classroom; defaults to 0

\item[\code{rel\_trans\_HH}] Relative attack rate of household contact (vs. classrom); defaults to 1

\item[\code{rel\_trans}] Relative attack rate of sustained contact (vs. classroom); defaults to 1/8

\item[\code{rel\_trans\_brief}] Relative attack rate of brief contact (vs. classroom); defaults to 1/50

\item[\code{p\_asymp\_adult}] Fraction of adults with asymptomatic (unsuspected) disease; defaults to 0.35

\item[\code{p\_asymp\_child}] Fraction of children with asymptomatic (unsuspected) disease; defaults to 0.7

\item[\code{attack}] Average daily attack rate in adults; defaults to 0.01

\item[\code{child\_trans}] Relative transmissibility of children (vs. adults); defaults to 1

\item[\code{child\_susp}] Relative transmissibility of children (vs. adults); defaults to 1

\item[\code{teacher\_trans}] Factor by which teacher transmissibility is reduced due to intervention; defaults to 1

\item[\code{teacher\_susp}] Factor by which teacher transmissibility is reduced due to intervention; defaults to 1

\item[\code{disperse\_transmission}] Whether transmission is overdispersed (vs. all have equal attack rate); default to T

\item[\code{isolate}] Whether symptomatic individuals isolate when symptoms emerge; defaults to T

\item[\code{dedens}] Whether dedensification measures reduce attack rate; defaults to F

\item[\code{run\_specials}] Whether special subjects are run; defaults to F

\item[\code{start}] Data frame from make\_class()
\end{ldescription}
\end{Arguments}
%
\begin{Value}
out data frame of child and teacher attributes.
\end{Value}
\inputencoding{utf8}
\HeaderA{make\_infected}{Set infection parameters}{make.Rul.infected}
%
\begin{Description}\relax
Set infection parameters for individuals infected at a particular timestep
\end{Description}
%
\begin{Usage}
\begin{verbatim}
make_infected(df.u, days_inf, set = NA, mult_asymp = 1, seed_asymp = F)
\end{verbatim}
\end{Usage}
%
\begin{Arguments}
\begin{ldescription}
\item[\code{set}] indication of seeding model vs. creating infections

\item[\code{mult\_asymp}] multiplier on asymptomatic infection; default is 1

\item[\code{seed\_asymp}] when making a seed, force to be asymptomatic; default is false

\item[\code{a}] id of infected individual

\item[\code{df}] school data frame from make\_school()
\end{ldescription}
\end{Arguments}
%
\begin{Value}
df.u with updated parameters
\end{Value}
\inputencoding{utf8}
\HeaderA{make\_schedule}{Make schedule}{make.Rul.schedule}
%
\begin{Description}\relax
Make a schedule of when individuals in the school community are
present/absent
\end{Description}
%
\begin{Usage}
\begin{verbatim}
make_schedule(time = 30, type = "base", total_days = 5, df)
\end{verbatim}
\end{Usage}
%
\begin{Arguments}
\begin{ldescription}
\item[\code{time}] number of days; defaults to 30

\item[\code{type}] "base", "On/off", "A/B"; defaults to "base"

\item[\code{total\_days}] number of days in school; defaults to 5

\item[\code{df}] data frame from make\_school()
\end{ldescription}
\end{Arguments}
%
\begin{Value}
d Returns a n x time data frame that indicates whether an individual is
in the school building at a particular time
\end{Value}
\inputencoding{utf8}
\HeaderA{make\_school}{Make school}{make.Rul.school}
%
\begin{Description}\relax
This function allows you to sort a synthetic population into classes.
It also assigns children to groups for alternating schedules and
ensures that children are in the same group as siblings.
It adds non-primary teacher staff, and if families are included, includes
two adult family members per child and one per adult staff member.
\end{Description}
%
\begin{Usage}
\begin{verbatim}
make_school(
  synthpop = synthMD,
  n_other_adults = 30,
  includeFamily = F,
  n_class = 4
)
\end{verbatim}
\end{Usage}
%
\begin{Arguments}
\begin{ldescription}
\item[\code{synthpop}] synthetic population; defaults to synthMD stored in file

\item[\code{n\_other\_adults}] Number of adults in the school other than primary teachers; defaults to 30

\item[\code{includeFamily}] whether to include family and adult family members of teachers, default = FALSE

\item[\code{n\_class}] number of classes per grade
\end{ldescription}
\end{Arguments}
%
\begin{Value}
out data frame of child and teacher attributes
\end{Value}
\inputencoding{utf8}
\HeaderA{mult\_runs}{Run model multiple times and summarize results}{mult.Rul.runs}
%
\begin{Description}\relax
Run model multiple times and summarize results
\end{Description}
%
\begin{Usage}
\begin{verbatim}
mult_runs(
  N = 500,
  n_other_adults = 30,
  n_contacts = 10,
  n_contacts_brief = 0,
  rel_trans_HH = 1,
  rel_trans = 1/8,
  rel_trans_brief = 1/50,
  p_asymp_adult = 0.35,
  p_asymp_child = 0.7,
  attack = 0.01,
  child_trans = 1,
  child_susp = 1,
  teacher_trans = 1,
  teacher_susp = 1,
  disperse_transmission = T,
  n_staff_contact = 0,
  n_HH = 0,
  n_start = 1,
  days_inf = 6,
  mult_asymp = 1,
  seed_asymp = F,
  isolate = T,
  dedens = 0,
  run_specials_now = F,
  time = 30,
  notify = F,
  test = F,
  test_sens = 0.7,
  test_frac = 0.9,
  test_days = NA,
  type = "base",
  total_days = 5,
  includeFamily = F,
  synthpop = synthMD,
  class = NA
)
\end{verbatim}
\end{Usage}
%
\begin{Arguments}
\begin{ldescription}
\item[\code{N}] number of runs

\item[\code{n\_other\_adults}] Number of adults in the school other than primary teachers; defaults to 30

\item[\code{n\_contacts}] Number of sustained contacts outside of the classroom; defaults to 10

\item[\code{n\_contacts\_brief}] Number of brief contacts outside of the classroom; defaults to 20

\item[\code{rel\_trans\_HH}] Relative attack rate of household contact (vs. classrom); defaults to 1

\item[\code{rel\_trans}] Relative attack rate of sustained contact (vs. classroom); defaults to 1/8

\item[\code{rel\_trans\_brief}] Relative attack rate of brief contact (vs. classroom); defaults to 1/50

\item[\code{p\_asymp\_adult}] Fraction of adults with asymptomatic (unsuspected) disease; defaults to 0.35

\item[\code{p\_asymp\_child}] Fraction of children with asymptomatic (unsuspected) disease; defaults to 0.7

\item[\code{attack}] Average daily attack rate in adults; defaults to 0.01

\item[\code{child\_trans}] Relative transmissibility of children (vs. adults); defaults to 1

\item[\code{child\_susp}] Relative transmissibility of children (vs. adults); defaults to 1

\item[\code{teacher\_trans}] Factor by which teacher transmissibility is reduced due to intervention; defaults to 1

\item[\code{teacher\_susp}] Factor by which teacher transmissibility is reduced due to intervention; defaults to 1

\item[\code{disperse\_transmission}] Whether transmission is overdispersed (vs. all have equal attack rate); default to T

\item[\code{n\_staff\_contact}] number of contacts a teacher/staff member has with other teachers/staff members

\item[\code{n\_HH}] number of households a household interacts with when not attending school; defaults to 0

\item[\code{n\_start}] number of infections to seed model; defaults to 1

\item[\code{days\_inf}] length of infectious period (assuming mild case or quarantined on symptoms)

\item[\code{mult\_asymp}] multiplier on asymptomatic infection; default is 1

\item[\code{seed\_asymp}] whether to seed with an asymptomatic case

\item[\code{isolate}] Whether symptomatic individuals isolate when symptoms emerge; defaults to T

\item[\code{dedens}] Whether dedensification measures reduce attack rate; defaults to F

\item[\code{time}] length of time to run model; defaults to 30

\item[\code{notify}] whether classrooms are notified and quarantined; defaults to F

\item[\code{test}] whether there is weekly testing; defaults to F

\item[\code{test\_sens}] test sensitivity; defaults to 0.7

\item[\code{test\_frac}] fraction of school tested; defaults to 0.9

\item[\code{test\_days}] vector indicating days on which students are tested; defaults to Sundays

\item[\code{type}] "base", "On/off", "A/B"; defaults to "base"

\item[\code{total\_days}] number of days in school; defaults to 5

\item[\code{includeFamily}] whether to include family, default = FALSE

\item[\code{synthpop}] synthetic population; defaults to synthMD

\item[\code{run\_specials}] Whether special subjects are run; defaults to F
\end{ldescription}
\end{Arguments}
\inputencoding{utf8}
\HeaderA{results}{Summarize multiple runs}{results}
%
\begin{Description}\relax
Summarize multiple runs
\end{Description}
%
\begin{Usage}
\begin{verbatim}
results(out)
\end{verbatim}
\end{Usage}
%
\begin{Arguments}
\begin{ldescription}
\item[\code{out}] output from mult\_runs
\end{ldescription}
\end{Arguments}
%
\begin{Value}
calc summarizes results from multiple runs
\end{Value}
\inputencoding{utf8}
\HeaderA{run\_care}{Set care-based transmission}{run.Rul.care}
%
\begin{Description}\relax
Determine who is infected at a timestep
from contact with an infected individual out of school
\end{Description}
%
\begin{Usage}
\begin{verbatim}
run_care(a, df, contacts)
\end{verbatim}
\end{Usage}
%
\begin{Arguments}
\begin{ldescription}
\item[\code{a}] id of infected individual

\item[\code{df}] school data frame from make\_school()

\item[\code{contacts}] graph of random contacts at time t
\end{ldescription}
\end{Arguments}
%
\begin{Value}
infs id of infected individuals
\end{Value}
\inputencoding{utf8}
\HeaderA{run\_class}{Set class transmission}{run.Rul.class}
%
\begin{Description}\relax
Determine who is infected at a timestep
in the same classroom as an infected individual
\end{Description}
%
\begin{Usage}
\begin{verbatim}
run_class(a, df)
\end{verbatim}
\end{Usage}
%
\begin{Arguments}
\begin{ldescription}
\item[\code{a}] id of infected individual

\item[\code{df}] school data frame from make\_school()
\end{ldescription}
\end{Arguments}
%
\begin{Value}
infs id of infected individuals
\end{Value}
\inputencoding{utf8}
\HeaderA{run\_household}{Set household transmission}{run.Rul.household}
%
\begin{Description}\relax
Determine who is infected at a timestep
in the same household as an infected individual
\end{Description}
%
\begin{Usage}
\begin{verbatim}
run_household(a, df)
\end{verbatim}
\end{Usage}
%
\begin{Arguments}
\begin{ldescription}
\item[\code{a}] id of infected individual

\item[\code{df}] school data frame from make\_school()
\end{ldescription}
\end{Arguments}
%
\begin{Value}
infs id of infected individuals
\end{Value}
\inputencoding{utf8}
\HeaderA{run\_model}{Run model}{run.Rul.model}
%
\begin{Description}\relax
Perform a single model run
\end{Description}
%
\begin{Usage}
\begin{verbatim}
run_model(
  time = 30,
  notify = F,
  test = F,
  test_days = NA,
  test_sens = 0.7,
  test_frac = 0.9,
  n_staff_contact = 0,
  n_HH = 0,
  n_start = 1,
  days_inf = 6,
  mult_asymp = 1,
  seed_asymp = F,
  df,
  sched
)
\end{verbatim}
\end{Usage}
%
\begin{Arguments}
\begin{ldescription}
\item[\code{time}] length of time to run model; defaults to 30

\item[\code{notify}] whether classrooms are notified and quarantined; defaults to F

\item[\code{test}] whether there is weekly testing; defaults to F

\item[\code{test\_sens}] test sensitivity; defaults to 0.7

\item[\code{test\_frac}] fraction of school tested; defaults to 0.9

\item[\code{n\_staff\_contact}] number of contacts a teacher/staff member has with other teachers/staff members; defaults to 0

\item[\code{n\_HH}] number of households a household interacts with when not attending school; defaults to 0

\item[\code{n\_start}] number of infections to seed model; defaults to 1

\item[\code{days\_inf}] length of infectious period (assuming mild case or quarantined on symptoms)

\item[\code{mult\_asymp}] multiplier on asymptomatic infection; default is 1

\item[\code{seed\_asymp}] whether to seed with an asymptomatic case

\item[\code{df}] school data frame from make\_school()

\item[\code{sched}] schedule data frame from make\_schedule()
\end{ldescription}
\end{Arguments}
%
\begin{Value}
df updated df with transmission results

time\_seed\_inf when the first individual was dropped in

class\_quarantine a matrix of class quarantine times

mat a check on if the people who you think are present are actually the ones present
\end{Value}
\inputencoding{utf8}
\HeaderA{run\_rand}{Set random transmission}{run.Rul.rand}
%
\begin{Description}\relax
Determine who is infected at a timestep
from random contact with an infected individual
\end{Description}
%
\begin{Usage}
\begin{verbatim}
run_rand(a, df, random_contacts)
\end{verbatim}
\end{Usage}
%
\begin{Arguments}
\begin{ldescription}
\item[\code{a}] id of infected individual

\item[\code{df}] school data frame from make\_school()

\item[\code{random\_contacts}] graph of random contacts at time t
\end{ldescription}
\end{Arguments}
%
\begin{Value}
infs id of infected individuals
\end{Value}
\inputencoding{utf8}
\HeaderA{run\_specials}{Set specials transmission}{run.Rul.specials}
%
\begin{Description}\relax
Determine who is infected at a timestep
from specials
\end{Description}
%
\begin{Usage}
\begin{verbatim}
run_specials(a, df, specials)
\end{verbatim}
\end{Usage}
%
\begin{Arguments}
\begin{ldescription}
\item[\code{a}] id of infected individual

\item[\code{df}] school data frame from make\_school()

\item[\code{specials}] classroom and teacher ids of specials at time t
\end{ldescription}
\end{Arguments}
%
\begin{Value}
infs id of infected individuals
\end{Value}
\inputencoding{utf8}
\HeaderA{run\_staff\_rand}{Set random staff transmission}{run.Rul.staff.Rul.rand}
%
\begin{Description}\relax
Determine who is infected at a timestep
from random contact between in-school adults
\end{Description}
%
\begin{Usage}
\begin{verbatim}
run_staff_rand(a, df, random_contacts)
\end{verbatim}
\end{Usage}
%
\begin{Arguments}
\begin{ldescription}
\item[\code{a}] id of infected individual

\item[\code{df}] school data frame from make\_school()

\item[\code{random\_contacts}] graph of random contacts at time t
\end{ldescription}
\end{Arguments}
%
\begin{Value}
infs id of infected individuals
\end{Value}
\inputencoding{utf8}
\HeaderA{synthMD}{Synthetic Maryland population}{synthMD}
\keyword{datasets}{synthMD}
%
\begin{Description}\relax
A data frame containing a synthetic population of children
ages 5-10, representative of the state of Maryland.
This is used by make\_class() to sort children into classes.
\end{Description}
%
\begin{Usage}
\begin{verbatim}
data(synthMD)
\end{verbatim}
\end{Usage}
%
\begin{Format}
A data frame with
\begin{description}

\item[HH\_id] household ID
\item[age] age
\item[flag\_mult] true if more than one child in the household, not used
\item[id] individual id \#

\end{description}

\end{Format}
%
\begin{Source}\relax
Wheaton, W.D., U.S. Synthetic Population 2010 Version 1.0 Quick Start Guide, RTI International, May 2014.
(\Rhref{https://fred.publichealth.pitt.edu/syn_pops}{website}).  Created with script demographic\_data2.R.
\end{Source}
\printindex{}
\end{document}
