% Options for packages loaded elsewhere
\PassOptionsToPackage{unicode}{hyperref}
\PassOptionsToPackage{hyphens}{url}
%
\documentclass[
]{article}
\usepackage{lmodern}
\usepackage{amssymb,amsmath}
\usepackage{ifxetex,ifluatex}
\ifnum 0\ifxetex 1\fi\ifluatex 1\fi=0 % if pdftex
  \usepackage[T1]{fontenc}
  \usepackage[utf8]{inputenc}
  \usepackage{textcomp} % provide euro and other symbols
\else % if luatex or xetex
  \usepackage{unicode-math}
  \defaultfontfeatures{Scale=MatchLowercase}
  \defaultfontfeatures[\rmfamily]{Ligatures=TeX,Scale=1}
\fi
% Use upquote if available, for straight quotes in verbatim environments
\IfFileExists{upquote.sty}{\usepackage{upquote}}{}
\IfFileExists{microtype.sty}{% use microtype if available
  \usepackage[]{microtype}
  \UseMicrotypeSet[protrusion]{basicmath} % disable protrusion for tt fonts
}{}
\makeatletter
\@ifundefined{KOMAClassName}{% if non-KOMA class
  \IfFileExists{parskip.sty}{%
    \usepackage{parskip}
  }{% else
    \setlength{\parindent}{0pt}
    \setlength{\parskip}{6pt plus 2pt minus 1pt}}
}{% if KOMA class
  \KOMAoptions{parskip=half}}
\makeatother
\usepackage{xcolor}
\IfFileExists{xurl.sty}{\usepackage{xurl}}{} % add URL line breaks if available
\IfFileExists{bookmark.sty}{\usepackage{bookmark}}{\usepackage{hyperref}}
\hypersetup{
  hidelinks,
  pdfcreator={LaTeX via pandoc}}
\urlstyle{same} % disable monospaced font for URLs
\usepackage[margin=1in]{geometry}
\usepackage{graphicx,grffile}
\makeatletter
\def\maxwidth{\ifdim\Gin@nat@width>\linewidth\linewidth\else\Gin@nat@width\fi}
\def\maxheight{\ifdim\Gin@nat@height>\textheight\textheight\else\Gin@nat@height\fi}
\makeatother
% Scale images if necessary, so that they will not overflow the page
% margins by default, and it is still possible to overwrite the defaults
% using explicit options in \includegraphics[width, height, ...]{}
\setkeys{Gin}{width=\maxwidth,height=\maxheight,keepaspectratio}
% Set default figure placement to htbp
\makeatletter
\def\fps@figure{htbp}
\makeatother
\setlength{\emergencystretch}{3em} % prevent overfull lines
\providecommand{\tightlist}{%
  \setlength{\itemsep}{0pt}\setlength{\parskip}{0pt}}
\setcounter{secnumdepth}{-\maxdimen} % remove section numbering
\usepackage[USenglish]{babel}
\usepackage{fancyhdr}
\pagestyle{fancy}
\renewcommand{\sectionmark}[1]{\markright{#1}}
\fancyhf{}
\lhead{{}}
\rhead{{\today}}
\cfoot{{\thepage}}
\usepackage[T1]{fontenc}
\usepackage{bm}
\usepackage{mathpazo}
\usepackage{lscape}
\usepackage{pdfpages}
\newcommand{\blandscape}{\begin{landscape}}
\newcommand{\elandscape}{\end{landscape}}
\usepackage{tabularx}
\usepackage{titlesec}
\usepackage{graphicx,xcolor}
\usepackage{wrapfig}
\usepackage{amssymb}
\usepackage{amsmath}
\usepackage{esint}
\usepackage{paralist}
\usepackage{outlines}
\newcommand{\I}{\textrm{I}}
\newcommand{\N}{\mathcal{N}}
\newcommand{\D}{\textrm{D}}
\newcommand{\E}{\mathbb{E}}
\setlength{\parskip}{1em}%0.5\baselineskip
\setlength{\parindent}{0pt}
\linespread{1.15}
\titleformat*{\section}{\Large\scshape\bfseries}
\titleformat*{\subsection}{\large\scshape\bfseries}
\titleformat*{\subsubsection}{\bfseries}
\titleformat*{\paragraph}{\bfseries}
\titleformat*{\subparagraph}{\bfseries}
\renewcommand{\thesection}{\Roman{section}.}%1.A.assubsections
\renewcommand{\thesubsection}{\Alph{subsection}.}%1.A.assubsections
\titlespacing{\section}{0pt}{2pt}{3pt}
\titlespacing{\subsection}{0pt}{2pt}{2pt}
\titlespacing{\subsubsection}{0pt}{0pt}{0pt}
\titlespacing{\paragraph}{0pt}{1pt}{5pt}
\titlespacing{\subparagraph}{10pt}{1pt}{5pt}
\usepackage{hyperref}
\usepackage[font={footnotesize}]{subcaption}
\usepackage[font={footnotesize}]{caption}
\usepackage{caption,setspace}
\captionsetup{font={stretch=1}}
\captionsetup[figure]{font=footnotesize,labelfont=footnotesize}
\usepackage{tabto}
\def\quoteattr#1#2{\setbox0=\hbox{#2}#1\tabto{\dimexpr\linewidth-\wd0}\box0}
\makeatletter
\newcommand{\pushright}[1]{\ifmeasuring@#1\hfill$\displaystyle#1$\fi\ignorespaces}
\makeatother
\newcommand{\FixMe}[1]{\textcolor{orange}{[#1]}}
\newcommand{\Comment}[1]{\textcolor{purple}{\textit{[#1]}}}
\newcommand{\Quickwin}{{\color{blue}{$\bigstar$}}}
\usepackage{letltxmacro}
\LetLtxMacro\Oldfootnote\footnote
\newcommand{\EnableFootNotes}{\LetLtxMacro\footnote\Oldfootnote}
\newcommand{\DisableFootNotes}{\renewcommand{\footnote}[2][]{\relax}}
\makeatother
\graphicspath{{../Output/"}}

\author{}
\date{\vspace{-2.5em}}

\begin{document}

\begin{figure}
\includegraphics{model2.png}
\caption{Model diagram.  The model includes 3 primary domains: households, schools, and out-of-school social/childcare mixing and incorporates a range of interventions to prevent or reduce transmission.}
\end{figure}

\clearpage

\blandscape

\begin{figure}
\centering
\includegraphics{Schools_figures_files/figure-latex/fig2-1.pdf}
\caption{Average number of total secondary transmissions over 30 days
(outside of the index case's household) following a single introduction
into a school community. These include both transmission directly from
the index case, as well as from secondary and tertiary cases. The top
panel shows elementary schools, where children are assumed to be less
susceptible and less infectious, while the bottom panel shows high
schools. Note that axes differ across rows. The x-axes varies the level
of prevention measure uptake, with low uptake assuming minimal
interventions and high uptake assuming intensive interventions. Line
colors correspond to scheduling strategies.}
\end{figure}

\elandscape

\clearpage

\begin{figure}
\centering
\includegraphics{Schools_figures_files/figure-latex/fig3.2-1.pdf}
\caption{Distribution of secondary transmissions when a single case is
introduced assuming self-isolation of individuals with clinical symptoms
(`Symptomatic isolation'). The y-axis displays the number of secondary
transmissions (outside of the index case's household) when a case is
introduced. Transmissions include both those directly from the index
case, as well as those from secondary and tertiary cases. Distributions
are truncated at the 99.5th quantile, i.e.~all outcomes occur with at
least probability 1/200.}
\end{figure}

\blandscape

\begin{figure}
\centering
\includegraphics{Schools_figures_files/figure-latex/unnamed-chunk-1-1.pdf}
\caption{Cumulative incidence over 8 weeks in elementary schools. The
x-axis shows the average daily community incidence per 100,000
population. The y-axis shows cumulative incidence over 8 weeks. Columns
denote different isolation, quarantine, vaccination, and detection
strategies, while rows show different population subgroups.}
\end{figure}

\elandscape

\blandscape

\begin{figure}
\centering
\includegraphics{Schools_figures_files/figure-latex/unnamed-chunk-2-1.pdf}
\caption{Cumulative incidence over 8 weeks in high schools. The x-axis
shows the average daily community incidence per 100,000 population. The
y-axis shows cumulative incidence over 8 weeks. Columns denote different
isolation, quarantine, vaccination, and detection strategies, while rows
show different population subgroups.}
\end{figure}

\elandscape

\end{document}
